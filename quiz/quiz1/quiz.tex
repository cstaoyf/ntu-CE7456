%\documentclass{...}

\input{./def/yf-formatting}

\usepackage{amsfonts, amsmath, amssymb, amsthm}
\usepackage{comment}
\usepackage{graphicx}
\usepackage{ifthen}
\usepackage{latexsym}
%\usepackage{times}
\usepackage[normalem]{ulem}

\input{./def/yf-def}

\def\dom{\prec}
\def\T{\mathcal{T}}

\newboolean{solver}\setboolean{solver}{true}
%\newboolean{solver}\setboolean{solver}{false}
\ifthenelse{\boolean{solver}}{\includecomment{sol}}{\excludecomment{sol}}

\begin{document}

\section*{CE7456: Quiz 1}

\noindent Name:  \hspace{50mm}
Student ID:

% \begin{center}
%     \uline{Classroom Discussion}
% \end{center}

\extraspacing


\extraspacing {\bf Problem 1 (25 marks).} Consider elements $e_1, e_2, ..., e_5$ whose weights are 1, 2, ..., 5, respectively. Build an alias structure on $S = \set{e_1, e_2, ..., e_5}$. Show the contents of each urn in the structure, and provide the necessary details to explain how each urn is obtained.


\pagebreak

\extraspacing {\bf Problem 2 (25 marks).} Consider the following tree $\T$, where the number in each node indicates the node's weight:
\begin{center}
    \includegraphics[height=50mm]{./artwork/ts}
\end{center}

\noindent Recall that, in tree sampling, we want to perform weighted sampling on the set of leaves of $\T$ (i.e., the gray nodes in the figure). Further recall that the tree sampling algorithm works by descending a path from the root, and may either {\em succeed} (in extracting a sample) or {\em fail}. Answer the questions below:

\vgap

(a) What is the success probability for one run of the algorithm?

(b) In expectation, how many repeats of the algorithm are needed to extract a sample?

\pagebreak

\extraspacing {\bf Problem 3 (25 marks).} Consider using logarithmic method to handle insertions. Recall that, at all times, the method partitions the current dataset $S$ into disjoint subsets $S_0, S_1, ..., S_h$ where $h = \lc \log_2 |S| \rc$ and each $S_i$ has size either 0 or $2^i$.

\vgap

Show the subset sizes when $|S| = 75$.

\pagebreak

\extraspacing {\bf Problem 4 (25 marks).} $S_1, S_2, ..., S_m$ are $m$ disjoint sets, where each element is an integer and is associated with an integer positive weight. Set $n = \max_{i=1}^m |S_i|$. On each $S_i$ ($1 \le i \le h$), there is
\myitems{
    \item an IQS structure that, given any interval $q$,  can extract a weighted sample from $q \cap S_i$ in $O(\log n)$ time;

    \item a ``range sum'' structure that, given any interval $q$, can obtain the total weight of the elements in $q \cap S_i$ in $O(\log n)$ time.
}
Given an arbitrary subset $U \subseteq \set{1, 2, ..., n}$, explain how to extract in $O(|U| \log n)$ time a weighted sample from $\bigcup_{j \in U} q \cap S_j$.


\end{document}
