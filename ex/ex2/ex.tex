%\documentclass{...}

\input{./def/yf-formatting}

\usepackage{amsfonts, amsmath, amssymb, amsthm}
\usepackage{comment}
\usepackage{graphicx}
\usepackage{ifthen}
\usepackage{latexsym}
%\usepackage{times}
\usepackage[normalem]{ulem}

\input{./def/yf-def}

\def\dom{\prec}
\def\T{\mathcal{T}}

\newboolean{solver}\setboolean{solver}{true}
%\newboolean{solver}\setboolean{solver}{false}
\ifthenelse{\boolean{solver}}{\includecomment{sol}}{\excludecomment{sol}}

\begin{document}

\section*{CE7456: Exercise List 2}
Prepared by Yufei Tao \\

% \begin{center}
%     \uline{Classroom Discussion}
% \end{center}

\boxminipg{0.95\linewidth}{
    These exercises are designed to help you practice the techniques taught in class. Attempt them ``at your own risk'' because they are research-oriented and can be rather challenging. Solutions will not be provided in written form. I will be happy to act as a collaborator and sounding board --- I look forward to discussing your creative approaches, refining your key steps, and walking through your attempted solutions with you.
}

\extraspacing {\bf Problem 1.} This problem is intended to guide you step by step in applying bootstrapping to design an IQS structure.

\vgap

Let $P$ be a set of $n$ points in $\real^2$, each carrying an integer positive weight. Given an axis-parallel rectangle $q$ and an integer $s \ge 1$, an IQS query returns $s$ independent weighted samples from $P \cap q$. Our goal is to design a data structure of linear space $O(n)$ that can answer such queries in $O(n^c + s)$ time, where $c$ can be an arbitrarily small constant.

\myitems{
    \item (a) Divide $\real^2$ into $\sqrt{n}$ slabs, each containing $\sqrt{n}$ points of $P$. An axis-parallel rectangle $q$ can be partitioned into
\myitems{
    \item a number of ``middle rectangles'', each spanning a slab in the x-projection;
    \item at most two ``boundary rectangles'', each of which has an x-projection contained within that of a slab.
}
Describe a data structure of $O(n)$ space such that, for each middle rectangle $q_{mid}$, the total weight of the points in $P \cap q_{mid}$ can be found in $O(\log n)$ time.

    \item (b) Describe a data structure of $O(n)$ space that can answer an IQS query in $O(\sqrt{n} \log n + s)$ time.

    Hint: For each middle rectangle, resort to a 1D IQS structure described in the class. For each boundary rectangle, simply read all the points in the corresponding slab.

    \item (c) Describe a data structure of $O(n)$ space that can answer an IQS query in $O(n^{1/3} \log n + s)$ time.

    Hint: Divide $\real^2$ into $n^{1/3}$ slabs, each containing $n^{2/3}$ points. You can no longer process each boundary rectangle in a brute-force manner. Resort to your structure in (b) instead.

    \item (d) Prove: Suppose that there is a structure of $O(n)$ space that guarantees IQS query time $O(n^{1/c} \log n + s)$ for a constant $c$. Then, there is a structure of $O(n)$ space that guarantees IQS query time $O(n^{1/(c+1)} \log n + s)$.

}

\extraspacing {\bf Problem 2.} Same problem as in Problem 1. Describe a structure of $O(n \polylog n)$ space that can answer an IQS query in $O(\polylog n + s)$ time.

\vgap

Hint: Talk to a LLM about ``range trees''.

\extraspacing {\bf Problem 3.} Let $P$ be a set of $n$ points in $\real^2$. Given an axis-parallel rectangle $q$, a {\em range query} returns $P \cap q$. In $O(n \log n)$ time, we can construct a kd-tree of $O(n)$ space that answers a range query in $O(\sqrt{n} + k)$ time, where $k = |P \cap q|$.

\myitems{
    \item (a) Prove: there is a semi-dynamic structure of $O(n)$ space that can answer a range query in $O(\sqrt{n} \log n + k)$ time and can support an insertion in $O(\log^2 n)$ amortized time.

    \item (b) Improve your solution to (a) by reducing the query time to $O(\sqrt{n} + k)$

    \vgap

    Hint: Analyze the time of the query algorithm of the logarithmic method more carefully --- you should see a geometric series.
}

\end{document}
